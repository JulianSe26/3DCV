\documentclass[conference, 11pt]{IEEEtran}
\IEEEoverridecommandlockouts

\usepackage{amsmath,amssymb,amsfonts}
\usepackage{algorithmic}
\usepackage{graphicx}
\usepackage{textcomp}
\usepackage{xcolor}
\usepackage{hyperref}
\usepackage{caption}
\usepackage[
    backend = biber,
    language = auto,
    style = numeric,
    sorting = none,
    block = space,
    hyperref = true,
    bibencoding = auto,
    giveninits = true,
    doi=false,
    isbn=false,
    alldates=short
]{biblatex}
\addbibresource{literature.bib}

\captionsetup{justification=centering,margin=0.5cm}

\def\BibTeX{{\rm B\kern-.05em{\sc i\kern-.025em b}\kern-.08em
    T\kern-.1667em\lower.7ex\hbox{E}\kern-.125emX}}
\begin{document}

\title{3DCV - Critical Driving Scenarios\\
{\small Generate critical driving scenarios in CARLA simulator and apply imitation learning to train a neural network}
}

\author{
    \IEEEauthorblockN{Christopher Klammt}
    \and
    \IEEEauthorblockN{Tobias Richstein}
    \and
    \IEEEauthorblockN{Julian Seibel}
    \and
    \IEEEauthorblockN{Karl Thyssen}
}

\maketitle

\begin{abstract}

\end{abstract}

\begin{IEEEkeywords}

\end{IEEEkeywords}

\section{Introduction}
Autonomous driving vehicles are a not just an idea for the more distant future, but rather a very current topic.
Not only Tesla, the company that dominates the news in this regard, is showing how far autonomous driving has come in the last years.
One crucial issue still lies in providing a stable and safe behavior in critical situations, especially in which traffic participants behave unexpectedly. 
The problem is that to robustly learn the appropriate behavior for these specific critical scenarios and to generalize using supervised learning a large amount of training data is needed.

In order to tackle this problem, our paper describes the development of a generator for such critical driving scenarios in the CARLA simulator.
These critical driving scenarios are mainly inspired by the report \citetitle{NHTSA:PreCrashScenarios} \cite{NHTSA:PreCrashScenarios} for the National Highway Traffic Safety Administration in the United States.
These contain scenarios such as avoiding an obstacle, lane changing with oncoming traffic or crossing traffic running a red light at an intersection (as shown in \autoref{fig:scenario-run_red_light}).

\begin{figure}[ht]
    \centering
    \includegraphics[width=0.7\linewidth]{figures/scenario-run_red_light.png}
    \caption{NHTSA scenario: Crossing traffic running a red light at an intersection \cite{CARLAChallenge:Scenarios}}
    \label{fig:scenario-run_red_light}
\end{figure}

Furthermore, after generating these different critical driving scenarios we utilize them to learn a robust model.
To do so imitation learning is used, in which an expert demonstrates the desired behavior in each critical situation and is then adopted by the model.

\section{Related Work}

\subsection{CARLA}
As presented by \citeauthor{Dosovitskiy17:CARLA}, CARLA is an open-source urban driving simulator for autonomous driving research \cite{Dosovitskiy17:CARLA}.
It enables handling different use cases within the general problem of driving, such as learning driving policies or training perception algorithms.
To control the simulation, e.g. changing weather, adding cars or pedestrians, an API is available in Python and C++.
CARLA consists of the simulator responsible for rendering etc. as well as multiple components, for example a traffic manager controlling the vehicles or a component handling the sensors.

\subsection{Critical driving scenarios}

Another component that works in unison with the CARLA simulator is the ScenarioRunner \cite{CARLA:ScenarioRunner}.
This is a module that makes it possible to define various traffic scenarios and execute them in the CARLA simulator.
These scenarios can be defined using Python or the OpenSCENARIO \cite{OpenScenario} standard.
The ScenarioRunner already contains some predefined Scenarios which are based on the critical driving scenarios as described in \citetitle{NHTSA:PreCrashScenarios} \cite{NHTSA:PreCrashScenarios}.

\subsection{Generating data in simulators}

In order to generate data it first is necessary to identify parameters that are to be adjusted and that should vary across the different entities.
The data generation can then be approached in different ways.
One possibility is to use a random statistical distribution of these parameters. 

Another way to go about data generation is to use learning-based methods to adjust the parameters of the simulator.
Such an approach was chosen by \citeauthor{DBLP:LearningToSimulate} and published in \citetitle{DBLP:LearningToSimulate} \cite{DBLP:LearningToSimulate}.
They proposed a reinforcement learning-based method to adjust the parameters of the synthesized data to maximize the accuracy of a model trained on that data.
A quite similar approach was chosen by \citeauthor{DBLP:Meta-Sim} \cite{DBLP:Meta-Sim} called Meta-Sim, which learns a generative model of synthetic scenes and modifies the attributes using a neural network.

\subsection{Imitation learning}

To learn a model based on labeled data some form of supervised learning is generally applied.
\citeauthor{Chen:LearningByCheating} propose a method called \citetitle{Chen:LearningByCheating} \cite{Chen:LearningByCheating}.
This is a two-stage method, in which first an agent is trained using privileged information.
In the second stage, the privileged agent acts as a teacher that trains a purely vision-based agent.

Another approach is \citetitle{Toromanoff_2020_CVPR} \cite{Toromanoff_2020_CVPR}. Here reinforcement learning is utilized to learn an optimal behavior policy based on a reward-function, effectively punishing wrong behavior such as leaving the track or running over pedestrians.

\printbibliography

\end{document}
